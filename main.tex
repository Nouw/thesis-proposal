\documentclass[stu]{apa7}
\usepackage{rotating} %for table orientation
\usepackage{caption}
\usepackage{graphics}  %rescaling tables

\usepackage{lscape}   % for table orientation
\usepackage{mathptmx}
\usepackage{soul}\setuldepth{article}
%\usepackage{times}
%\normalfont
%\usepackage[T1]{fontenc}
%\usepackage[mtplusscr,mtbold]{mathtime}
% For citations
\usepackage{apacite}            % Load the APA citation package
\bibliographystyle{apacite}      % Set the bibliography style to APA
% For links
\usepackage{hyperref}
\usepackage[dvipsnames,svgnames,table]{xcolor}%
% just for fun: make nice colours for the links (default is kind of nasty)
%\hypersetup{colorlinks=true,linkcolor=DarkBlue,urlcolor=Indigo}
\hypersetup{colorlinks=true, linkcolor=blue, citecolor=blue, urlcolor=magenta}
%\usepackage[colorlinks=true, linkcolor=blue, citecolor=blue, urlcolor=magenta]{hyperref}
% for graphics
\usepackage{graphicx}

\usepackage{plantuml}
\usepackage{tikz}
\usetikzlibrary{arrows}
\usepackage{aeguill}
\usepackage{adjustbox} % for resizing the plantuml graphics
\usepackage{dot2texi}
\usepackage{booktabs}
% for revering to section names
\usepackage{nameref}
% package for creating colored boxes
\usepackage{tcolorbox}
% for seperating sections into sub files
\usepackage{subfiles}
% Some extra typesetting for equations
\usepackage{mathtools}
% some maths shit
\usepackage{amssymb,amsmath,amsthm}
% For file including
\usepackage{import}
\usepackage[pdf]{graphviz}
% Math things
\newtheorem{theorem}{Theorem}
\newtheorem{definition}{Definition}
% For citetation
\usepackage{dirtytalk}
\begin{document}
% TODO: Update title
    \title{LLM for CBRN scenario generation}
    \author{Fabio Dijkshoorn}
    \affiliation{Utrecht University}
    \course{Bachelor Thesis}
    \duedate{24th February 2025}
    \professor{Prof. dr. ir. Jan Broersen}
    \maketitle

    \section{Introduction}\label{sec:introduction}
    Personal Protective Equipment (PPE) is a critical component of workplace safety,
    designed to protect individuals from a wide range of hazards, including chemical,
    biological, radioactive and nuclear (CBRN) threats.
    For civil first responders, proper PPE usage is not merely a recommendation; incorrect
    use can significantly elevate health risks and cause operational delays ~\cite{world2020personal}.
    For instance, wearing multiple layers of gloves can significantly reduce dexterity,
    prolonged use of PPE can lead to heat stress, and the viewing foils of
    face shields (depending on the suit system) can cause light reflection and
    refraction, making observation difficult and contributing to eye fatigue.

    Selecting the appropriate PPE for a given scenario is a complex task, primarily due
    to the complicating factor that higher protection usually introduces higher
    physiological burden.
    Furthermore, it also requires not only a thorough understanding of the hazard, but
    also an interpretation of regulatory standards and an in-depth knowledge of
    material properties.
    To aid in the selection of PPE, there exist models to simulate the physiological
    burden ~\cite{brasser2023scenario}.
    However, these models often rely on accurate scenario description and parameter
    definition as input.
    This dependency makes them labor-intensive and limits their adaptability in
    rapidly evolving threat environments.

    Recent advancements in scenario generation using Large Language Models (LLMs) such as
    \textit{LLMScenario}~\cite{chang_llmscenario_2024} have shown promising results in the
    realm of autonomous vehicles, but LLMs have also shown competence in addressing challenges
    in the radioactive and nuclear domain~\cite{iob_nuclear_2024}.
    Yet their potential to support PPE selection in bio-toxin incidents remains uncharted.

    Although current models offer valuable insights, their reliance on precise inputs
    hinders their ability to adapt to dynamic bio-toxin scenarios.
    To bridge this gap, this paper proposes an integrated framework that leverages
    the generative capabilities of LLMs to augment mathematical simulation models.
    Specifically, it explores the question: How can Large Language Models (LLMs) be
    integrated with mathematical simulation models to enhance adaptability
    in PPE selection for dynamic bio-toxin incidents?

    \section{Methodology}\label{sec:methodology}
    To realize the potential of the framework, a multi-stage process will be employed
    to integrat expert input with LLM scenario generation.
    The system transformers unstructured expert narratives into structured
    physiological parameters through a hybrid human-AI pipeline.

    In the initial phase, a chat-based interface collects detailed hazard description
    from domain experts.
    For example, an expert might describe a scenario as follows, \say{a 30-year-old
    male is running from location A to B, while wearing a PPE suit with protection level A\ldots}
    Such narratives may include environmental factors (e.g., temperature, humidity),
    subject parameters (e.g., age, fitness, weight), and task-specific details (e.g.,
    workload intensity, duration).

    Following data collection, the raw input is transformed to align with the
    physiological burden model as described by Brasser ~\cite{brasser2023scenario}.
    This transformation is the core of the framework, as it leverages the LLM not only
    to reformat but also to augment the provided information by identifying and
    filling missing details.

    To ensure the generated scenarios are both comprehensive and plausible,
    retrieval-augmented generation (RAG) is utilized.
    This technique integrates relevant external information into the model's outputs, effectively grounding the
    generated scenario in established data sources
    ~\cite{lewis2021retrievalaugmentedgenerationknowledgeintensivenlp}.

    Enhancing the LLM's performance further, two advanced prompting techniques will be
    evaluated.
    \textit{Few-Shot Prompting}, this method embeds several carefully curated examples
    into the system prompt to guide the LLM\@.
    Few-shot prompting enables the LLM to recognize nuanced patterns in expert narratives.
    The inclusion of examples improves the model's ability to generate context-aware
    and accurate outputs ~\cite{brown2020languagemodelsfewshotlearners}.
    \textit{Chain-of-Thought (CoT) prompting}, explicitly instructs the model to break
    down the task into a series of intermediate reasoning steps before arriving at the
    final output.
    By deconstructing the scenario generation process, this approach minimizes error
    propagation and enhances logical consistency.
    Forcing the model to articulate its reasoning not only increases transparency but
    also improves robutness when handling complex and ambiguous inputs
    ~\cite{wei2023chainofthoughtpromptingelicitsreasoning}.

    The language model of choice is Meta's LLaMA-3.2 family.
    LLaMA was selected primarily because of its open-source licensing, which allows
    for the freedom to customize.
    Furthermore, its modular and extensible architecture facilitates seamless
    integration with RAG and prompting strategies.
    Therefore, making it ideally suited for research environments that demand
    adaptability and scalability ~\cite{grattafiori2024llama3herdmodels}.

    Through this hybrid approach, the proposed framework is designed to produce robust,
    detailed and contextual rich scenarios that meet the requirements of the
    physiological burden model and the fast-changing environment of bio-toxin hazards.

    \section{Relevance}\label{sec:relevance}

    This research is conducted as part of the European Union's Horizon-funded project
    EMBRACE (Grant Agreement ID: 101168322), which aims to enhance Europe's response to
    bio toxin incidents.
    Within this context, the proposed framework directly addresses the need for scalable,
    adaptive tools to optimize PPE selection in dynamic threat environments.

    The integration of LLMs with mathematical simulation models aligns with the project's
    focus on leveraging AI-driven technologies to improve situational awareness and
    operational decision-making.
    By automating scenario generation and parameter optimization,
    this work contributes to lightening the cognitive burden on civil first responders and
    making it possible to developer training scenarios quicker.

% citation
    \bibliography{references}
\end{document}

