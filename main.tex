\documentclass[stu]{apa7}
\usepackage{rotating} %for table orientation
\usepackage{caption}
\usepackage{graphics}  %rescaling tables

\usepackage{lscape}   % for table orientation
\usepackage{mathptmx}
\usepackage{soul}\setuldepth{article}
%\usepackage{times}
%\normalfont
%\usepackage[T1]{fontenc}
%\usepackage[mtplusscr,mtbold]{mathtime}
% For citations
\usepackage{apacite}            % Load the APA citation package
\bibliographystyle{apacite}      % Set the bibliography style to APA
% For links
\usepackage{hyperref}
\usepackage[dvipsnames,svgnames,table]{xcolor}%
% just for fun: make nice colours for the links (default is kind of nasty)
%\hypersetup{colorlinks=true,linkcolor=DarkBlue,urlcolor=Indigo}
\hypersetup{colorlinks=true, linkcolor=blue, citecolor=blue, urlcolor=magenta}
%\usepackage[colorlinks=true, linkcolor=blue, citecolor=blue, urlcolor=magenta]{hyperref}
% for graphics
\usepackage{graphicx}

\usepackage{plantuml}
\usepackage{tikz}
\usetikzlibrary{arrows}
\usepackage{aeguill}
\usepackage{adjustbox} % for resizing the plantuml graphics
\usepackage{dot2texi}
\usepackage{booktabs}
% for revering to section names
\usepackage{nameref}
% package for creating colored boxes
\usepackage{tcolorbox}
% for seperating sections into sub files
\usepackage{subfiles}
% Some extra typesetting for equations
\usepackage{mathtools}
% some maths shit
\usepackage{amssymb,amsmath,amsthm}
% For file including
\usepackage{import}
\usepackage[pdf]{graphviz}
% Math things
\newtheorem{theorem}{Theorem}
\newtheorem{definition}{Definition}
\begin{document}
% TODO: Update title
\title{LLM for CBRN scenario generation}
\author{Fabio Dijkshoorn}
\affiliation{Utrecht University}
\course{Bachelor Thesis}
\duedate{14 February 2025}
\professor{Unknown}
\maketitle
\section{Introduction}\label{sec:introduction}
Personal Protective Equipment (PPE) is a critical component of workplace safety,
designed to protect individuals from a wide range of hazards, including chemical,
biological, radioactive and nuclear (CBRN) threats.
For civil first responders, proper PPE usage is not merely a recommendation; incorrect
use can significantly elevate health risks and cause operational delays ~\cite{world2020personal}.
%TODO: Add example of bio-toxin event

Selecting the appropriate PPE for a given scenario is a complex task that requires
extensive domain expertise.
It requires not only a thorough understanding of the hazard, but also an interpretation
of regulatory standards and an in-depth knowledge of material properties.
While state-of-the-art models exist to aid in PPE selection, they often rely on
accurate scenario description and parameters definition as input.
This dependency makes them labor-intensive and limits their adaptability in rapidly
evolving threat environments.

Recent advancements in scenario generation using Large Language Models (LLMs) such as
\textit{LLMScenario}~\cite{chang_llmscenario_2024} have shown promising results in the
realm of autonomous vehicles, but LLMs have also shown competence in addressing challenges
in the radioactive and nuclear domain~\cite{iob_nuclear_2024}.
Yet their potential to support PPE selection in bio-toxin incidents remains uncharted.

Although current models offer valuable insights, their reliance on precise inputs
hinders their ability to adapt to dynamic bio-toxin scenarios.
To bridge this gap, this paper proposes an integrated framework that leverages the
generative capabilities of LLMs and the precision of mathematical simulation models to automate and
optimize PPE selection in hazardous bio-toxin scenarios.

\section{Methodology}\label{sec:methodology}
To realize the potential of the framework, a multi-stage process will be used to
seamlessly integrate expert input, advanced LLM-based scenario generation and
evaluation of the mathematical output.
The process is divided into two primary phases, Expert Input Augmentation and Outcome
analysis.

\subsection{Expert Input Augmentation}\label{subsec:expert-input-augmentation}
In the initial phase, a chat-based interface allows domain experts to input critical
data such as weather conditions, biometric information, location and workload
parameters.
This expert provided data forms the basis for generating realistic hazard scenarios.
The expert's input is then augmented by the LLM, by prompts.
Two prompts techniques will be evaluated for augmenting the data, Chain-of-Thought ~\cite{wei2023chainofthoughtpromptingelicitsreasoning}
and Few-Shot prompting ~\cite{brown2020languagemodelsfewshotlearners}.
Additionally, Retrieval-Augmented Generation (RAG) ~\cite{lewis2021retrievalaugmentedgenerationknowledgeintensivenlp}
is utilized to incorporate relevant external information, ensuring that the generated
scenarios are both comprehensive and plausible.

\subsection{Outcome Analysis}\label{subsec:outcome-analysis}
TODO

\section{Relevance}\label{sec:relevance}

This research is conducted as part of the European Union's Horizon-funded project
EMBRACE (Grant Agreement ID: 101168322), which aims to enhance Europe's response to
bio toxin incidents.
Within this context, the proposed framework directly addresses the need for scalable,
adaptive tools to optimize PPE selection in dynamic threat environments.

The integration of LLMs with mathematical simulation models aligns with the project's
focus on leveraging AI-driven technologies to improve situational awareness and
operational decision-making.
By automating scenario generation, parameter optimization, and result interpretation,
this work contributes to lightening the cognitive burden on civil first responders and
making it possible to developer training scenarios quicker.

% citation
\bibliography{references}
\end{document}

